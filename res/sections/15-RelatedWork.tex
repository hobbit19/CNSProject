\section{Related work}
As the popularity of Bitcoins and altcoins grow, the number of work on
de-anonymization their transactions increases.
There are mainly 2 kind of approaches.

The fist method exploits the structure of peers and connection on which the
blockchain relies to attack anonymity of transactions. In~\cite{anonBitcoin}
the authors propose a system in which they bind IP addresses to transaction to
define the ownership of transactions. Another valuable study
is~\cite{deanonP2P}. In the latter authors propose an attack to Bitcoin
infrastructure that they claim to work even in challenging scenarios (e.g.
bypassing firewalls and NATs) with a few number of computers. We do not follow
this idea because we do not want to bind IP addresses to transaction, but
physical person to transaction. Another problem is that this kind of approach
could be computationally heavy. The other technique, instead, has no
requirements on the computational power.

An orthogonal approach, that is the one that we followed, is to search for
public information on the Web in order to find out wallets that are advertised
by the owner and then infer other related wallets. This approach is used
in~\cite{anonAnalysis}: the authors retrieved public available addresses from
Bitcoin dedicated forums in order to discover related wallets. This approach is
explained with a graph view of the transaction occurred in the blockchain:
addresses are the vertexes and transactions instead are the edges. In this
graph, called \textit{user graph} addresses inputs of the same transaction are
clustered in a single node because owned by the same person. This privacy
problem was for the first time issued by Satoshi Nakamoto and comes out
because people do not generate new wallets for each transaction as he
suggested~\cite{satoshi}. Another work based on that idea is~\cite{fistful}.
These works are pretty old (they are done 2011) and they focus only on Bitcoins
and in retrieving addresses from Bitcoin forums. A possible issue is the
multi-signature native support of Bitcoins. This requires that a transaction,
to be valid, is signed with more than a single public
key~\cite{multisignature}: in this scenario all this addresses will be grouped
in the same cluster. A more fine grained study of this cluster will probably
find out the differences, if integrated with different techniques. We must
consider that multi-signature usually application involve closely related
people~\cite{multisignaturebitcoinwiki}, so for certain application can be
usefull to group this people together.

Other related studies work on transaction profiling. In~\cite{fullDiscl} is
proposed a profiling transaction model in the Bitcoin environment. The
simplest example is the following: if we have 2 addresses and the first one
sends periodically the same amount of Bitcoins in US Dollars, then probably
the person who ones the second wallet work for the first one.

In this research we do not consider de-anonymization in scenarios in which are
used \textit{mixing} services. These kind of services are used to augment
transacions anonymity: basically, a users sends money to these services that
are mixed and gave back to the initial user generating new addresses. These
anonymizing techniques are principally used in illegal purchase or because
that moneys are involved in some sort of crimes, the most important is money
laundering~\cite{laudering}.
