\section{Related work}
As the popularity of Bitcoins and altcoins growth, the number of work on
de-anoymize their transactions increases. There are mainly 2 kind of approaches.

The fist method exploit the structure of peers and connection on which the
blockchain relies to attack anonymity of transactions. In~\cite{anonBitcoin}
the authors propose a system in which they bind IP addresses to transaction to
define the ownership of transactions. Another valuable study
is~\cite{deanonP2P}. In the latter authors propose an attack to Bitcoin
infrastructure that they claim to work even in challenging senarios (e.g.
bypassing firewalls and NATs) with a few number of computers. We do not follow
this idea because we do not want to bind IP addresses to transaction, but
physical person to transaction. Another problem is that this kind of approach is
computationally heavy, has no requirements on the computational power.

An orthogonal, as the one we followed, is to search for public information on
the Web in order to find out wallets that are advertised by the owner and then
infer other related wallets. This approach is used in~\cite{anonAnalysis}: the
authors retrieved public available addresses from Bitcoin dedicated forums in
order to discover related wallets. This approach is explained with a graph view
of the transaction occured in the blockchain: addresses are vertex and
transactions instead are edges. In this graph, called \textit{user graph}
addresses inputs of the same transaction are clustered in a single node because
owned by the same person. This privacy problem was for the first time issued by
Satoshi Nakamoto and comes out because people do not generate new wallets for
each transaction as suggested~\cite{satoshi}. Another work based on that idea
is~\cite{fistful}. This work are pretty old and they focus only on Bitcoins and
in retrieving addresses from Bitcoin forums. Another possible issue is the
multi-signature native support of Bitcoins. This requires that a transaction is
signed with more than a sigle public key~\cite{multisignature}: in this
scenario all this addresses will be grouped in the same cluster. A more fine
grained study of this cluster could find out the differences, but most of
multisignature application involve closely related
people~\cite{multisignaturebitcoinwiki}.

Other related studies work on transaction profiling. In~\cite{fullDiscl} are proposed some profiling of transaction model in the Bitcoin environment. The simplest example is the following: if we have 2 addresses and the first one sends periodically the same amount of Bitcoins in US Dollars, then problably the person who ones the second wallet work for the first one.