\section{Related work}
As the popularity of Bitcoins and altcoins growth, the number of work on
de-anoymize their transactions increases. There are mainly 2 kind of approaches.

The fist one exploit the structure of peers and connection on which the blockchain relies to attack anonymity of transactions. In~\cite{anonBitcoin} the authors propose a system in which they bind IP addresses to transaction to define the ownership of transactions. Another valuable study is~\cite{deanonP2P}. In the latter authors propose an attack to Bitcoin infrastructure that they claim to work even in challenging senarios (e.g. bypassing firewalls and NATs) with a few number of computers.

An orthogonal, as the one we followed, is to search for public information on the Web in order to find out wallets that are advertised by the owner and then infer other related wallets. This approach is used in~\cite{fistful}