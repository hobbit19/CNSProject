\section{Background} \label{background}
\subsection{Blockchain and Bitcoin}
Bitcoin was the the first widely adopted non-proprietorial virtual
currency~\cite{bib:respCrypto} that makes use of the
Blockchain technology. The Blockchain of Bitcoin rely on digital
signatures to prevent the repudiation of a transaction, without a third
party check~\cite{bib:anonAnalysis}.
To achieve this, it was designed a peer-to-peer payment system,
in which each peer maintains a copy of the complete transaction
history~\cite{bib:fistful}. 
In bitcoin and its derivatives a transaction is a move of value 
from transaction inputs to transaction outputs.
These inputs come from previous unspent transaction outputs (UTXO).
UTXOs can be redeemed (i.e. spent) if a user can satisfy the locking
script contained in the transaction output, also known as \emph{encumbrance}.
Usually the encumbrance locks the output to a specific bitcoin address,
allowing the transfer of certain amount to the address' owner.
Transactions can have multiple inputs and multiple 
outputs~\cite{bib:bitcoin:mastering}. 
To abstract from this low-level view we could say that the bitcoin
address $A$ sends a specific amount of bitcoins to address $B$, 
if there is a transaction output that can be redeemed from $A$ that
is input of a transaction output, that can be redeemed by $B$.
Fortunately, all APIs used throughout our work already provide this
abstraction.

Once a transaction is built and signed by the
sender, it is broadcasted. Some nodes of the bitcoin network, called
\textit{miners}, collect transactions into blocks and try to find a difficult
proof-of-work for the block~\cite{bib:hashcash, bib:pricing}, i.e. they
need to find a nonce such that the hash of the block contains a certain number
of leading zeros. Since this operation is
computationally difficult the successful generation of a block is well
rewarded.~\cite{bib:satoshi}. If one of the miners succeeds it broadcasts the
block to the network. The remaining nodes check if the transactions of the
block are valid and whether they were already spent or not. 

One of the key factors of the success of the blockchain technology
is the prevention of double spending problem without relying on a third party,
e.g. a bank. This is achieved thanks to transparent
transactions~\cite{bib:bitcoinbeyond}.

The bitcoin system, also, preserves user's personal privacy, giving them some
sort of pseudo-anonymity. In this system every user owns at least one wallet,
that is associated with a private key and a public one, also known as address.
The latter are 26-35 alphanumeric characters~\cite{bib:bitcoinwiki:address}
\footnote{We do not consider addresses in \texttt{Bech32} format. Their use is
discouraged until more support will be available~\cite{bib:bitcoinwiki:bech32}}
long and are indispensable in order to be able to exchange Bitcoins.
In fact to transfer Bitcoin to another person you have to know her
public key, that works as user account. No personal data is required to generate
the keys. It is as well possible to send money to non-existent addresses: when
someone will create a new wallet with that address she could redeem
them. Each user could generate more wallets, indeed it is recommended
to generate a new address for each transaction to better preserve the
anonymity~\cite{bib:satoshi}. In order to use a certain address the owner of
a wallet must conserve carefully the private key associated with it, in fact
if it is lost, there is no possibility to recover it~\cite{bib:respCrypto}.


\subsection{Altcoins}
After the success of Bitcoin other digital currencies took place.
All cryptocurrencies different from Bitcoin are known as 
\emph{altcoins} (Alternative Coins)~\cite{bib:bitcoinbeyond}.
One of the most famous is
\textbf{Ethereum}. The peculiar characteristic of this cryptocurrency is
the support for a Turing-Complete language, that allows developers to
write their own smart-contracts and decentralized
applications~\cite{bib:ethereum:whitepaper}.
Plenty of aspects are very similar to Bitcoin: even
this altcoin is based on blockchain, transactions are public, users are
pseudo-anonymous. In Ethereum the public keys associated with wallets are 20
bytes number encoded as hexadecimal numbers (i.e. 40 ciphers). Usually
the public key is prepended with the characters
\texttt{0x}~\cite{bib:ethereum:whitepaper}.
Two other altcoins that have public transactions and were considered
throughout our work are: \textbf{Litecoin} and \textbf{Dogecoin}.
The first one is a cryptocurrency born in 2011 with the aim to improve
Bitcoin and thought to be ``lighter'' to create blocks.
They changed the proof of work algorithm to permits the
generation of a new block every 2.5 minutes rather than 10 minutes as in
bitcoin~\cite{bib:litecoin:wiki}.
Litecoin wallets starts with \texttt{L} or \texttt{M} followed by an
alphanumeric string whose length is between 24 and 36 characters.
The latter is an altcoin derived from Litecoin, born as a joke in 2013
but that rapidly increased its market capitalization, exceeding 1
billion dollars in January 2018~\cite{bib:dogecoin:marketcap}.
Dogecoin addresses starts with \texttt{D} followed by 33 alphanumeric
characters. A slightly different altcoin is \textbf{Monero}: this currency was
created with the aim of privacy, so the transaction chain is not public
and there is no way to know how much money a wallet has collected and
with which other wallets it exchanged transactions.
Monero public addresses are 90 characters alphanumeric addresses that starts
with \texttt{4}.


\subsection{Pseudo-anonymity in Bitcoin system}
We can define a subject as \textit{anonymous} if the subject is not identifiable
within a set of subjects, denoted as the anonymity
set~\cite{bib:terminology}. Bitcoin system does not provide it
because~\cite{bib:deanon}
\begin{enumerate*}[label=\roman*),itemjoin={,\quad}]
\item the real-name authentication mechanism helps Bitcoin service
providers to find the addresses that ever deposited and withdrew
\item a Bitcoin wallet advertised on Internet can be related to its
owner
\item the chain of transactions is transparent
\item exchanging inputs from an address to another one could expose the
sender
\item the change address of transactions could be classified by
attackers to the sender.
\end{enumerate*}
So in that context we can speak about pseudo-anonymity: until a user
does not associate personal information with a certain wallet, she
could be considered anonymous.
