\section{Background}
\subsection{Blockchain and Bitcoin}
Bitcoin was the first criptocurrency developed in order to use Blockchain
technology. Bitcoin and Blockchain rely on the digital signatures
to prevent the repudiation of a transaction, without a third party check.
To achieve this, it was designed a peer-to-peer payment system, in which each
peer maintain a copy of the complete transaction history. Every transaction is
hashed, validated and stored in blocks connected in a list. These are created
by \textit{miners}, some members of the network, with a process called
\textit{mining}. The Blockchain history is agreed upon using a proof-of-work
system~\cite{pricing}~\cite{hashcash}, so each new block generated is rewarded.
Blockchain, as already stated, was designed to deal specifically with
Bitcoin. This system becomes so popular because it allows
to prevent double spending problem using transparent transactions. It
preserves user personal privacy, giving them some sort of pseudo-anonymity.
In this system every user owns at least one wallet, that is associated with an
address. Addresses are 26-35 alphanumeric characters long and are indispensable
in order to be able to exchange Bitcoins. In fact to give Bitcoin to another
person you have to know her wallet address, that works as user account. It is
also possible to send money to non-existent addresses: when someone will create
a new wallet with that address it could redeem them. Each user could generate
more wallets and they do not require any kind of authentication. Often it is
recommended to generate a new address to each transaction, in order to better
preserve the anonymity\cite{satoshi}. In order to use a certain address the
owner of the wallet must know the private key associate with it, in fact if it
will be lost there is no possibility to recover it. 


\subsection{Altcoins}
After the success of Bitcoin other digital currencies take place. All these
cryptocurrencies that are different from Bitcoin are called \textit{altcoins}
(Alternative Coins)~\cite{bitcoinbeyond}. One of the most famous is 
\textbf{Ethereum}. A characteristics of this cryptocurrency is that in ... a
fork happened and we have two different version of this coin: \textbf{Ethereum
Classic} and \textbf{Ethereum}. Lots of aspects are similar to Bitcoin: even
this altcoin is based on blockchain, transactions are public, users are
pseudo-anonymous. In Ethereum wallets are associated with 40 alphanumeric
characters addresses that could has as prefix \texttt{0x}. Three other altcoins
that have public transactions and were considered throughout our work are: 
\textbf{Bitcoin Cash}, \textbf{Litecoin} and \textbf{Dogecoin}. The first one
is a hard fork of Bitcoin performed in 2017. Addresses format are the same of
the original coin. The second one, instead, is a criptocurrency born in 2011
with the aim to improve Bitcoin and thought to be ``lighter'' to create blocks.
Litecoin wallets starts with \texttt{L} or \texttt{M} followed by an
alphanumeric string whose length is between 24 and 36 characters. The latter
is an altcoin born as a joke in 2013 but that rapidly increased its value,
exceeding 1 billion dollars in January 2018. Dogecoin addresses starts with
\texttt{D} followed by 33 alphanumeric characters. A slightly different altcoin
is \textbf{Monero}: this currency was created with the aim of privacy, so the
transaction chain is not public. Monero wallets are a 90 alphanumeric addresses
that starts with \texttt{4}.


\subsection{Pseudo-anonymity in Bitcoin system}
We can define a subject as anonymous if subject is not identifiable within a
set of subjects, denoted as the anonymity set~\cite{terminology}. Bitcoin system
does not provide it because~\cite{deanon}
\begin{enumerate*}[label=\roman*),itemjoin={,\quad}]
\item The real-name authentication mechanism helps Bitcoin service providers to
find the addresses that ever deposited and withdrew
\item a Bitcoin wallet advertised on Internet can be related to its owner
\item the chain of transactions is transparent
\item exchanging inputs from an address to another one could expose the sender
\item the change address of transactions could be classified by attackers to
the sender.
\end{enumerate*}
So in that context we can speak about pseudo-anonymity: until a user does not
associate personal information with a certain wallet, she could be considered
anonymous.