\section{Results}
After we implemented and tested \texttt{Nduja} we run it. The first step of the
algorithm give us ?!? addresses. Finally, after the clustering, we increase
this number to ?!? wallets. In Figure~?!? is depicted the increase of wallet we
found out. In Figure~?!? we can see how the addresses that we collected are
distributed in the different currencies. As we aspected the number of Monero
and Ethereum wallets are lower than the other: Monero has private transactions,
so it is not possible to analyze its blockchain, instead Ethereum does not
allow multiple-input transactions, so it is not possible to cluster wallets as
for the others. The number of Monero wallets could be so low even because it is
an altcoin not so famous and it was born in order to preserve users privacy, so
people could be more aware to advertise it. Addresses are connected ?!?
accounts, so most of them own more than one wallet, on average ?!? addresses
per person. As we expected the currency on which we could retrieve more data is
Bitcoin yet. Analyzing data collected we discover that for ?!? accounts we are
able to connect a Twitter, Facebook or Linkedin account. The least social
network is the most insteresting, because people share more sensitive data on
that because of its aim. In Figure~?!? this data are represented. Finally, ?!? accounts in our database are strongly related with at least another one: this means that we have discovered more than an account for a certain person or that are people that are in some way related.