\section{Nduja}
%~ Here we could explain our idea. We could compare it with previous works.
%~ Add evaluation if we implement something
To bind physical people to wallet addresses we exploit the fact that
people frequently advertise their own public keys to receive founds
through  donations.

We build our data set by searching onto Twitter and on developers'
repositories (e.g. github).


% TWITTER
On the former platform there are several people that ask for donations 
for disparate purposes and sometimes the owners of some twitter accounts
decide to donate some altcoins to gain followers and popularity (e.g., with hashtag \texttt{\#BTCGiveAway}). They advertise 
their intent with a tweet, in which they ask the readers to follow
them and reply with their own altcoin wallet. After some time the
owner draw from the replies one of the authors and performs the donation.

% REPOSITORIES
On the latter source free-time developers that distribute their
own software with free licenses may ask the users to donate some
altcoins to guarantee the further development of their tool. 

Both these resources provide well-documented search APIs~\cite{}~\cite{}.

\subsection{Methodology}
\label{sec:methodology}
We define a set of altcoins that we would like to de-anonymize. For each of them we define the symbol, the name and a regular expression
that matches the wallet addresses for a particular coin.

Based on this information we build queries as explained in section~\ref{sec:queries}.

\subsection{Queries} 
\label{sec:queries}

To maximize the number of results


