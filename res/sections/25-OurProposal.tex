\section{Our proposal}
\label{proposal}
%~ Here we could explain our idea. We could compare it with previous works.
%~ Add evaluation if we implement something

\subsection{Data retrieving}
Data that we used to implement our idea is publicly available on the Web. In
fact to bind physical people to wallet addresses we exploit the fact that
people frequently advertise their own public keys to receive founds
through donations.

We build our data set by searching onto social networks and on developers'
repositories (e.g. github) and then expanding it with different techniques
explained in other studies~\cite{bib:fistful}. Following our idea, after
discovering a wallet we could certainly associate it with an account that could
have some other public information. Because a person could have more than a
single wallet we search for all transactions in which the wallet is involved
and we suppose that all the addresses input of a certain transaction are under
the control of the same person \cite{bib:satoshi} \cite{bib:deanon}
\cite{bib:fistful}.

\subsubsection*{Twitter}
We have chosen this social network because most \textit{tweets} of the users are
public. There are several people that ask for donations for the most disparate
purposes and, furthermore the owners of some accounts decide to donate some
Bitcoins or altcoins to gain followers and popularity (e.g., using hashtag
\texttt{\#BTCGiveAway}). They advertise their intent with a tweet, in which
they ask the readers to follow them and reply with their own altcoin wallet.
After some time the owner draw from the replies one of the authors and performs
the donation.

\subsubsection*{Repositories}
Free-time developers that distribute their own software with free licenses may
ask the users to contribute to their project with some Bitcoins or altcoins to
guarantee the further development of their tools. The choice of targeting our
research even on developers is because they are more likely to follow new
technologies and proposal such as new cryptocurrencies. 

\subsection{Cryptocurrencies we looked for}
\label{subsec:currencies}
The cryptocurrencies on which we focused our search are: Bitcoin, Ethereum,
Litecoin, Dogecoin and Monero. This search could be easily expanded on every
other altcoins knowing the pattern of the wallets' addresses. We focused our
efforts on these because we have decide to choose three of the most famous
cryptocurrencies (Bitcoin, Ethereum and Litecoin), Dogecoin because even if
it was born as a joke, it reached a really high value. Finally, we even
search for Monero addresses because it is a currency that does not allow to
access to transaction chain, so we could only check how many users we could
discover that advertise their Monero addresses, estimating the rate of use of
it and this could give us how much information is lost trying to de-anonymize
currencies with private transactions.

\subsection{Methodology}
\label{sec:methodology}
After we have decided the set of cryptocurrencies that we would like to
de-anonymize, listed in section~\ref{subsec:currencies}, we have defined for
each of them the symbol (3 or 4 characters that identify a particular
currency), the name and a regular expression that matches the addresses of a
certain coin.

Based on this information we build queries as explained in
section~\ref{sec:queries}.
