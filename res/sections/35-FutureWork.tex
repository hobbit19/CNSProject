\section{Future work} \label{future}
This project, as we said before, is a proof of concept and it could only gives
an indication of how much information you cold retrieve. A simple improvement
is to re-launch our program using premium keys (in particular for Twitter) and
searching for other altcoins that we have not considered yet. Another possible
improvement will be to increase the number of websites in which \texttt{Nduja}
makes its research: cryptocurrency dedicated forums or other social network,
e.g. Reddit, in which there is a section dedicated to this topic. Another
enhancement is to implement some heuristics to group Ethereum wallets, since in
this altcoins it is not possible to have multiple-inputs transaction. These
heuristics could take advantage of the fact that in Ethereum it is
not possible to open new wallet for each transaction because there is an
explicit mapping from accounts to their states~\cite{bib:ethersok}.
improvements are related to data mining: an important possibility could be
performing some analysis on account related to addresses to discover, if
possible, related account on other social networks~\cite{bib:osinference}.
Another possible improvement is to implement some sort of more sophisticated
pre-processing to discover as much services as possible, grouping all addresses
used from the same addresses in a single cluster. This could speed-up research,
discarding a priori a huge amount of wallets. A possible related future work is
profiling users and transactions: once the graph of clusters is created it is
possible to study the edges (an edge between two clusters $C$, $D$ is added
if there is a transaction from a wallet in $C$ and a wallet in $D$), disclosing
possible remuneration profiling (as described in~\cite{bib:fullDiscl}) or
profiling people associated with clusters. A simple example could be this: if
some user donate some money to an open source project, probably she uses it and
if the project has some peculiarities you could infer some information on the
donator.