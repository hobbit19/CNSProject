% As a general rule, do not put math, special symbols or citations
% in the abstract
Many cryptocurrencies have come into existence in recent years, with Bitcoin
the most famous among them. One of the most attracting features of these
currencies is the pseudo-anonymity: the owner of a wallet remains unknown until 
it cannot be associated to her personal information. 
This property is essential to preserve users' privacy, but it could be exploited to hide criminal activities. There is a large number of works that present techniques to preserve or augment the address privacy or to de-anonymize them.
In this paper we analyze a well-known 
privacy threat that exploits the multiple input transactions and the inexperiency of some users, that allow to bind wallets' addresses to personal
data of their owner.
To pursue this aim we firstly build a database of wallets bounded
to people, exploiting information publicly available on the internet, e.g.
advertized public keys for donations. 
Secondly we elaborate the obtained results to correlate known accounts with other accounts, by taking advantage of
the facts that the transaction history is public and that users do not change
their wallets at each transaction as suggested in the best practices.

We finally claim that some techniques described in the literature are not
still valid, because the obtained results are not fully trustful and some
correlations are spurious, probably due to the spread of mixing services and 
users' awareness.





