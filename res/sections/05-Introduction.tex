\section{Introduction}
Blockchain was introduced the first time in 2008, by Satoshi Nakamoto, an
anonymous programmer (actually, it is probably a group of people) that
designed it with the aim to create a virtual and decentralized currency,
Bitcoin~\cite{satoshi}. Since then, this technology has been developed and
studied in different fields than the financial one, for managing documents, in
the notary public, as decentralized storage and so on~\cite{air}. Bitcoin grew
rapidly and lots of other cryptocurrencies came out: lots of them are really
similar to Bitcoin, and try to improve some aspects of that system. Some
altcoins try to make the system ''lighter''~\cite{bib:litecoin:wiki}, some other try to improve the
privacy~\cite{bib:zercoin:white_paper, bib:zerocash:white_paper, 
bib:monero:white_paper}, some other are only thought as an 
alternative~\cite{bib:dogecoin}. 
The success of these
currencies could be traced back to the possibility to generate wallets 
without any work,
their anonymity (or pseudo-anonymity) and because people are every day becoming
more familiar with this new technologies.

Alongside blockchain and its application, different kinds of attacks have
been developed. These malicious behavior principally focus on the
deanonymization of Bitcoin (or other currencies) wallets: in layman's terms 
it means to find out who owns a certain wallet. A typology of attacks 
concentrate on the fact that Bitcoin and Blockchain rely on Internet
infrastructure and try to deduce the identity of a certain wallet address
user by the IP address~\cite{deanon}.
Another kind of approach is to concentrate on transaction graph and infer data,
for example to find out if wallets are owned by the same person~\cite{fistful}.

In this paper we exploit the fact that some people do not follow the common
rules in order to preserve the anonymity of their wallets by associating to them
their personal data, without knowing that they risk to compromise even the other
wallets owned by them.


\subsection{Paper structure}
This paper is organized as follow: Section II...