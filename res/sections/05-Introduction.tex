\section{Introduction}
Blockchain was introduced the first time in 2008, by Satoshi Nakamoto, an
anonymous programmer (actually, it is probably a group of people) that
designed it with the aim to create a virtual and decentralized currency,
Bitcoin~\cite{bib:satoshi}.
Thenceforward this technology has been developed and
studied in fields different than the financial one: for managing documents, in
the notary public, as decentralized storage and so on and so
forth~\cite{bib:air}. Bitcoin grew
rapidly and a large number of other cryptocurrencies came out:
a good deal of them are really
similar to Bitcoin, and try to improve some aspects of that system. Some
altcoins try to make the system ''lighter''~\cite{bib:litecoin:wiki},
some other try to improve the privacy~\cite{bib:zerocoin:white_paper,
bib:zerocash:white_paper,
bib:monero:white_paper}, some other enhance the blockchain technology with a
Turing-complete language to enforce smart-contracts'
execution~\cite{bib:ethereum:whitepaper}.
The success of these
currencies could be traced back to the possibility to join the network easily,
generate wallets in a simple manner,
their anonymity (or pseudo-anonymity) and also because people are
becoming more familiar with these new technologies.


Alongside blockchain and its application, different kinds of deanonymization
attacks have been developed.
These attacks principally focus on the
deanonymization of Bitcoin (or other currencies) wallets: in layman's terms
it means to find out who owns a certain wallet.
A typology of attacks exploits the fact that Bitcoin and Blockchain rely
on Internet infrastructure and try to deduce the identity of a certain wallet
address user by the IP address~\cite{bib:deanon}.
Another kind of approach consists in the analysis of the transaction graph
to infer data, e.g. to find out if some wallets are owned by the same
person~\cite{bib:fistful}.
In this paper we exploit the fact that some people do not follow the common
rules in order to preserve the anonymity of their wallets by associating them
with their personal data, without knowing that they risk to compromise even
other wallets owned by them.


\subsection{Paper structure}
This paper is organized as follows:~\autoref{background} gives a brief
overview of the Blockchain technology, Bitcoin, altcoins and pseudo-anonymity.
\autoref{relworks} evaluates the related works and compare them
with our idea. Our proposal and its implementation are described
in~\autoref{proposal} and~\autoref{nduja} respectively.
The obtained results are showed in~\autoref{results} and the issues
are presented in ~\autoref{sec:issues}.
Finally we sketch the proposals for future work in~\autoref{future} and draw
our conclusion and possible enhancements in~\autoref{conclusion}.
